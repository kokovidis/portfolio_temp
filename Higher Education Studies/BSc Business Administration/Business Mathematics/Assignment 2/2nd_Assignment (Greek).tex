\documentclass[12pt,a4paper]{article}
\usepackage[utf8x]{inputenc}
\usepackage{ucs}
\usepackage{tabularx}
\usepackage{paralist}

\usepackage[greek]{babel}
\usepackage{amsmath}
\usepackage{amsfonts}
\usepackage{amssymb}
\usepackage{makeidx}
\usepackage{graphicx}
\usepackage[left=2cm,right=2cm,top=2cm,bottom=2cm]{geometry}

\author{Κοκοβίδης Συμεών 61/13}
\title{ Τμήμα Οργάνωσης \& Διοίκησης επιχειρήσεων \\ Μαθηματικά για Διοίκηση Επιχειρήσεων Ι \\ Δεύτερη εργασία στο \LaTeX}
\setlength{\parindent}{0pt} % gia na min dimiourgei esoxes ka8e fora pou allazw 
\everymath{\displaystyle} % gia na min mikrainei tipota
\begin{document}
\maketitle 
\textbf{{\Large 1 \ Ασκήσεις πάνω στα οικονομικά}}\\\\
\textbf{Άσκηση 10}\\\\

Η συνάρτηση ζήτησης και προσφοράς ενός αγαθού δίνονται από τους τύπους $Q_d = 150-50lnP$ και $Q_s = 50+ 50lnP$ , όπου $P, Q$ η τιμή και η ποσότητα του προϊόντος.\\

\begin{enumerate}
\item Να προσδιοριστεί το κοινό πεδίο ορισμού των συναρτήσεων δεδομένου οτι η συνάρτηση ζήτησης πρέπει να είναι φθίνουσα και η συνάρτηση προσφοράς αύξουσα.\\
\item Να υπολογιστει το σημείο ισορροπίας της αγοράς.\\
\item Με βάση τη συνάρτηση ζήτησης να υπολογιστει η συνάρτηση εσόδων και να εξετάσετε αν τα έσοδα μεγιστοποιούνται στο σημείο ισορροπίας της αγοράς.\\
\item Να υπολογιστεί το πλεόνασμα του παραγωγού.\\
\item Να υπολογιστεί το πλεόνασμα ενός καταναλωτή που είναι διατεθιμένος να πληρώσει έως \euro $e^2$\\
\end{enumerate} 
\textbf{Λύση}\\
\begin{enumerate}
\item Θα πρέπει να είναι $Q_d> 0, Q_s>0 ,$ $Q_d' <0 $ , $Q_s'>0$ και $P>0$ (I) \\
Η παράγωγος συνάρτηση της $Q_d$ είναι η $Q_d'=-\frac{50}{P} $ και της $Q_s$ είναι η $Q_s'=\dfrac{50}{P}$ \\\\
Επομένως θα είναι $-\dfrac{50}{P}<0 \Leftrightarrow -50<P \Leftrightarrow P>-50$ (II),\\ και $ \dfrac{50}{Ρ}>0\Leftrightarrow 50>P \Leftrightarrow P<50 $ (III) \\

Επίσης πρέπει $150-50lnP > 0 \Leftrightarrow 150 > 50 lnP \Leftrightarrow 3 >lnP \Leftrightarrow e^3 > P \Leftrightarrow P< e^3$ \textlatin{(IV)} \\ οπως και $ 50+50lnP >0 \Leftrightarrow 50lnP > -50 \Leftrightarrow lnP > -1 \Leftrightarrow P > e^{-1} $ \textlatin{(VI)} 

Το κοινό πεδίο ορισμού προκύπτει απο τους περιορισμούς Ι, ΙΙ, ΙΙΙ, \textlatin{IV} , \textlatin{VI} 
και επομένως θα είναι: \\ $ P \in  (e^{-1}, e^3) $\\\\
\item Το σημείο ισορροπίας βρισκέται με την εξίσωση των συναρτήσεων ζήτησης και προσφοράς δηλαδή: $Q_d=Q_s \Leftrightarrow 150-50lnP = 50+ 50lnP \Leftrightarrow 100 = 100lnP \Leftrightarrow lnP = 1 \Leftrightarrow P=e $\\\\
Αντικαθιστώντας σε μία απο τις δύο εξισώσεις βρίσκουμε οτι $Q_d|  = Q_s|  = 100$ οπου $Q_d|  , Q_s| $ η ποσότητα ζήτησης και προσφοράς στην ισορροπία.\\

\item Η συνάρτηση εσόδων, στη γενική της μορφή είναι: $TR = P \cdot Q $\\\\
Με δεδομένη την συνάρτηση  ζήτησης , τότε η συνάρτηση ζήτησης θα είναι: $TR = P \cdot Q_d \Rightarrow (150-50lnP)P \Rightarrow 150P - 50PlnP $\\\\
Για να βρούμε που μεγιστοποιούνται τα έσοδα θα πρέπει να χρησιμοποιήσουμε το κριτήριο της δεύτερης παραγώγου και επομένως,

$TR' = (150P - 50PlnP)' = 150 - 50P \dfrac{1}{P} + 50lnP = 150 - 50 + 50lnP = 100 + 50lnP  $ , \\

$TR'=0 \Rightarrow 100+ 50lnP =O \Rightarrow lnP = -2 \Rightarrow P= e^{-2} $\\

Που σήμαινει οτι μοναδικό και πιθανό ακρότατο (μέγιστο ή ελάχιστο) είναι η τιμή $P= e^{-2} $ και επομένως η τιμή ισορροπίας δεν μεγιστοποιεί τα κέρδη.\\

\item Το πλεόνασμα του παραγωγού δίνεται απο τον τύπο :

 $ PS=\int\limits_0^{p|} {{q^S}} (p)dp (q) $ όταν ανεξάρτητη μεταβλητή είναι η τιμή, \\με $p|$  την τιμή στην ισορροπία της αγοράς.\\\\

Στη προκειμένη περίπτωση αντικαθιστώντας προκύπτει: \\

$PS=\int\limits_0^e {{\rm{(5}}0 + {\rm{5}}0{\rm{lnp)}}dp{\rm{ }}}  = \int\limits_0^e {{\rm{5}}0dp + {\rm{ }}\int\limits_0^e {{\rm{5}}0{\rm{lnpdp}}} {\rm{ }}}  = [50{\rm{p}}]_0^e + 50\int\limits_0^e {{\rm{lnpdp}}}  = [50{\rm{p}}]_0^e + 50\int\limits_0^e {{\rm{(p)'lnpdp}}}  = [50{\rm{p}}]_0^e + 50[{\rm{plnp  -  }}\int\limits_0^e {{\rm{p(lnp)'dp}}} ] = [50{\rm{p}}]_0^e + 50[{\rm{plnp}} - p]_0^e = 50e - 0 + 50(e - e) - 50(0 - 0) = 50e$\\\\

\item To πλεόνασμα του ενός καταναλωτή ανάλογα με το εφικτό ποσό που μπορει να διαθέσει προκύπτει απο τον τύπο: $CS = \int\limits_{p|}^{(\varepsilon \varphi \iota \kappa \tau o)} {{q^D}} (p)dp$\\
Αντικαθιστώντας προκύπτει: \\

$CS = \int\limits_e^{{e^2}} {(150 - 50\ln p)} dp = [150p]_e^{{e^2}} - 50\int\limits_e^{{e^2}} {\ln p} dp = [150p]_e^{{e^2}} - 50([p\ln p]_e^{{e^2}} - \int\limits_e^{{e^2}} {p(\ln p} )'dp]) = [150p]_e^{{e^2}} - [50p\ln p]_e^{{e^2}} + [50p]_e^{{e^2}} = 150{e^2} - 150e - 50{e^2}\ln {e^2} + 50e\ln e + 50{e^2} - 50e = 200{e^2} - 50{e^2}\ln {e^2} - 150e = 200{e^2} - 100{e^2}\ln e - 150e = 100{e^2} - 150e$\\\\
\end{enumerate}
%--------------------------------------------19--------------%

\textbf{Άσκηση 19}\\

Η συνάρτηση των εσόδων μιας επιχείρησης που παράγει 2 προϊόντα, Α και Β, είναι η: \\


\begin{center}
$TR(q_A,q_B) = -4q_A^2 - 3q_B^2 + 6q_A q_B + 20 q_A - 120 $
\end{center}


'Οπου $q_A,q_B$ οι εβδομαδιαίες ποσότητες παραγωγής των 2 προϊόντων σε τόνους και $R$ τα εβδομαδιαία έσοδα σε χιλιάδες \euro.\\

\begin{enumerate}
\item Να βρεθεί το επίπεδο παραγωγής που μεγιστοποιεί τα έσοδα της επιχείρησης, καθώς και τα μέγιστα κέρδη.\\
\item Να βρεθεί το επίπεδο παραγωγής που μεγιστοποιεί τα έσοδα αν οι εβδομαδιαίες ποσότητες παραγωγής πρέπει να ικανοποιούν τον περιορισμό:$10q_A-6q_B+200=0$
\end{enumerate}


\textbf{Λύση}\\
\begin{enumerate}
\item Θα βρούμε τις μερικές παραγώγους: \\\\
 
$TR{q_A} = {{\partial TR} \over {\partial {q_A}}} =  - 8{q_A} + 6{q_B} + 20$ , \ \  
$TR{q_B} = {{\partial TR} \over {\partial {q_B}}} =  - 6{q_B} + 6{q_A}$\\\\

Άρα το ανάδελτα της \textlatin{TR} είναι 

\[\nabla TR = \left\{ {\begin{array}{*{20}{c}}
{ - 8{q_A} + 6{q_B} + 20}\\
{ - 6{q_B} + 6{q_A}}
\end{array}} \right\}\]

Βρίσκοντας και τις δεύτερες μερικές παραγώγους αλλα και παραγωγίζοντας την μερική παράγωγο της μιας ως προς την άλλη μεταβλήτη. \\
\[TR{q_A}{q_A} = \frac{{{\partial ^2}TR}}{{\partial {q_A}^2}} =  - 8{\rm{ }}{\rm{, \       }}TR{q_B}{q_B} = \frac{{{\partial ^2}TR}}{{\partial {q_B}^2}} =  - 6, \ {\rm{      }}TR{q_A}{q_B} = \frac{{{\partial ^2}TR}}{{\partial {q_A}\partial {q_B}}} = 6\]
Προκείπτει και η εσσιανή:

\[HTR = \left( {\begin{array}{*{20}{c}}
{\frac{{{\partial ^2}TR}}{{\partial {q_A}^2}}}&{\frac{{{\partial ^2}TR}}{{\partial {q_A}\partial {q_B}}}}\\
{\frac{{{\partial ^2}TR}}{{\partial {q_A}\partial {q_B}}}}&{\frac{{{\partial ^2}TR}}{{\partial {q_B}^2}}}
\end{array}} \right) = \left( {\begin{array}{*{20}{c}}
{ - 8}&6\\
6&{ - 6}
\end{array}} \right)\] \\

\begin{center}
Εξισώνοντας το ανάδελτα της \textlatin{TR} με μηδέν (Συνθήκες Πρώτης Τάξης)
\end{center}

\[\nabla TR = 0 \Leftrightarrow \left\{ {\left. {\begin{array}{*{20}{c}}
{ - 8{q_A} + 6{q_B} + 20 = 0}\\
{ - 6{q_B} + 6{q_A} = 0}
\end{array}} \right|} \right. \Rightarrow \left\{ {\left. {\begin{array}{*{20}{c}}
{ - 8{q_A} + 6{q_B} + 20 = 0}\\
{6{q_A} = 6{q_B}}
\end{array}} \right|} \right. \Rightarrow \left\{ {\left. {\begin{array}{*{20}{c}}
{ - 2{q_A} =  - 20}\\
{6{q_A} = 6{q_B}}
\end{array}} \right|} \right.\; \Rightarrow \]


\[ \Rightarrow \left\{ {\left. {\begin{array}{*{20}{c}}
{2{q_A} = 10}\\
{6{q_A} = 6{q_B}}
\end{array}} \right|} \right. \Rightarrow \left\{ {\left. {\begin{array}{*{20}{c}}
{{q_A} = 10}\\
{{q_B} = 10}
\end{array}} \right|} \right.\] \\

\begin{center}
Και βρίσκοντας εάν η εσσιάνη ειναι αρνητική μέσω των ηγετικών κύριων ελάσσονων:\\

$|H_1| =  - 8$ \ \ και $|Η_2| = [(-8) \cdot (-6)] - 6 \cdot 6= 12$, που όντως είναι.\\
\end{center}

Καταλήγουμε πώς ο συνδιασμός απο 10 τόνους του προϊοντος Α και 10 τόνους του Β, οδηγει στα μέγιστα έσοδα για την επιχείρηση. Αντικαθιστώντας λοιπόν στην αρχική συνάρτηρη εσόδων \textlatin{TR}, $TR(10,10) = -4\cdot10^2 - 3\cdot10^2 + 6\cdot10 \cdot 10 + 20 \cdot 10 - 120 = - 20$, καταλήγουμε πως ποτέ δεν έχει έσοδα αρα και ποτέ δεν εχει κέρδη.\\
\item Αφού έχουμε περιορισμό, θα δημιουργήσουμε την συνάρτηση \textlatin{Lagrange}: \\
\begin{center}
$L({q_A},{q_{B,}}\lambda ) =  - 4q_A^2 - 3q_B^2 + 6{q_A}{q_B} + 20{q_A} - 120 - \lambda ({\rm{1}}0{q_A} - {\rm{6}}{q_B} + {\rm{2}}00)$ 
\end{center} 
θα βρούμε το ανάδελτα της \textlatin{L} , των συνδιασμό λύσεων του συστημάτος: $\nabla L = 0 $ και έπειτα μέσω της εσσιανής της \textlatin{L} για το ποιος συνδιασμός λύσεων αποτελεί και τοπικό μέγιστο. Όταν έχω δύο μεταβλητές και έναν περιορισμό στη συνάρτηση \textlatin{Lagrange}, αν η ορίζουσα της \textlatin{HL} $({x_{o,}}{y_o},\lambda )$ είναι θετική τότε έχω τοπικό μέγιστο.
\end{enumerate}
\newpage
%--------------------------------------------5--------------%
\textbf{{\Large 2 \ Γενικές ασκήσεις}}\\\\
\textbf{Άσκηση 5}\\\\
Δίνεται ο πίνακας:
\[{\rm A} = \left( {\begin{array}{*{20}{c}}
3&1&2\\
1&4&{ - 3}\\
5&{ - 2}&7
\end{array}} \right)\]

\begin{enumerate}
\item Να υπολογιστει ο πίνακας $Α^2$
\item Να εξετάσετε αν τα διανύσματα στήλες του πίνακα Α είναι γραμμικά ανεξάρτητα.
\item Να λυθεί το σύστημα $Αx=\textbf{0}$ όπου  \textbf{0} = ( 0 0 0 )'\\\\
\end{enumerate}
\textbf{Λύση}\\
\begin{enumerate}
\item 
\[A^2=A \cdot A = \left( {\begin{array}{*{20}{c}}
3&1&2\\
1&4&{ - 3}\\
5&{ - 2}&7
\end{array}} \right)\left( {\begin{array}{*{20}{c}}
3&1&2\\
1&4&{ - 3}\\
5&{ - 2}&7
\end{array}} \right) = \]\\\\
\[ = \left( {\begin{array}{*{20}{c}}
{3 \cdot 3 + 1 \cdot 1 + 2 \cdot 5}&{3 \cdot 1 + 1 \cdot 4 + 2 \cdot ( - 2)}&{3 \cdot 2 + 1 \cdot ( - 3) + 2 \cdot 7}\\
{1 \cdot 3 + 4 \cdot 1 + ( - 3) \cdot 5}&{1 \cdot 1 + 4 \cdot 4 + ( - 3) \cdot ( - 2)}&{1 \cdot 2 + 4 \cdot ( - 3) + ( - 3) \cdot 7}\\
{5 \cdot 3 + ( - 2) \cdot 1 + 7 \cdot 5}&{5 \cdot 1 + ( - 2) \cdot 4 + 7 \cdot ( - 2)}&{5 \cdot 2 + ( - 2) \cdot ( - 3) + 7 \cdot 7}
\end{array}} \right) = \]\\\\
\[ = \left( {\begin{array}{*{20}{c}}
{9 + 1 + 10}&{3 + 4 - 4}&{6 - 3 + 14}\\
{3 + 4 - 15}&{1 + 16 + 6}&{2 - 12 - 21}\\
{15 - 2 + 35}&{5 - 8 - 14}&{10 + 6 + 49}
\end{array}} \right) = \left( {\begin{array}{*{20}{c}}
{20}&3&{17}\\
{ - 8}&{23}&{ - 31}\\
{48}&{ - 17}&{65}
\end{array}} \right)\]
\\\\


\item Για να είναι τα διανύσματα στήλες του πίνακα, γραμμικά ανεξάρτητα θα πρέπει οι στήλες του πίνακα ως διανύσματα: \\

\[{A_1} = \left( {\begin{array}{*{20}{c}}
3\\
1\\
5
\end{array}} \right),{A_2} = \left( {\begin{array}{*{20}{c}} \ 
1\\
4\\
{ - 2}
\end{array}} \right),{A_3} = \left( {\begin{array}{*{20}{c}} \ 
2\\
{ - 3}\\
7
\end{array}} \right)\]

να δίνουν ως μοναδικη λύση το \textbf{ 0} , στο παρακάτω σύστημα:
$\kappa {A_1} + \lambda {A_2} + \mu {A_3} = 0$\\\\

Άρα έχουμε 
\[\kappa \left( {\begin{array}{*{20}{c}}
3\\
1\\
5
\end{array}} \right) + \lambda \left( {\begin{array}{*{20}{c}}
1\\
4\\
{ - 2}
\end{array}} \right) + \mu \left( {\begin{array}{*{20}{c}}
2\\
{ - 3}\\
7
\end{array}} \right) = 0 \Rightarrow \left( {\begin{array}{*{20}{c}}
{3\kappa  + \lambda  + 2\mu }\\
{\kappa  + 4\lambda  - 3\mu }\\
{5\kappa  - 2\lambda  + 7\mu }
\end{array}} \right) = 0 \Rightarrow \left. {\begin{array}{*{20}{c}}
{3\kappa  + \lambda  + 2\mu  = 0}\\
{\kappa  + 4\lambda  - 3\mu  = 0}\\
{5\kappa  - 2\lambda  + 7\mu  = 0}
\end{array}} \right\}\]

\[\left. {\begin{array}{*{20}{c}}
{\lambda  =  - 3\kappa  - 2\mu }\\
{\kappa  + 4\lambda  - 3\mu  = 0}\\
{5\kappa  - 2\lambda  + 7\mu  = 0}
\end{array}} \right\}\left. {\begin{array}{*{20}{c}}
{\lambda  =  - 3\kappa  - 2\mu }\\
{\kappa  - 12\kappa  - 8\mu  - 3\mu  = 0}\\
{5\kappa  + 6\kappa  + 4\mu  + 7\mu  = 0}
\end{array}} \right\}\left. {\begin{array}{*{20}{c}}
{\lambda  =  - 3\kappa  - 2\mu }\\
{ - 11\kappa  - 11\mu  = 0}\\
{11\kappa  + 11\mu  = 0}
\end{array}} \right\}\left. {\begin{array}{*{20}{c}}
{\lambda  =  - 3\kappa  - 2\mu }\\
{ - 11\kappa  = 11\mu }\\
{11\kappa  =  - 11\mu }
\end{array}} \right\}\]

\[\left. {\begin{array}{*{20}{c}}
{\lambda  =  - 3\kappa  - 2\mu }\\
{\kappa  = \mu }
\end{array}} \right\}\left. {\begin{array}{*{20}{c}}
{\lambda  =  - 3\mu  - 2\mu }\\
{\kappa  = \mu }
\end{array}} \right\}\left. {\begin{array}{*{20}{c}}
{\lambda  =  - 5\mu }\\
{\kappa  = \mu }
\end{array}} \right\}(\kappa ,\lambda ,\mu ) = (\mu , - 5\mu ,\mu )\]
\begin{center}
Επομένως τα διανύσματα, δεν είναι γραμμικά ανεξάρτητα.
\end{center}
\item 
\[{\rm{Ax}} = 0 \Rightarrow \left( {\begin{array}{*{20}{c}}
3&1&2\\
1&4&{ - 3}\\
5&{ - 2}&7
\end{array}} \right)x = 0 \Rightarrow \left( {\begin{array}{*{20}{c}}
{3{x_1}}&{1{x_2}}&{2{x_3}}\\
{1{x_1}}&{4{x_2}}&{ - 3{x_3}}\\
{5{x_1}}&{ - 2{x_2}}&{7{x_3}}
\end{array}} \right) = \left( {\begin{array}{*{20}{c}}
0&0&0
\end{array}} \right)' \Rightarrow \]\\

\[ \Rightarrow \left( {\begin{array}{*{20}{c}}
{3{x_1}}&{1{x_2}}&{2{x_3}}\\
{1{x_1}}&{4{x_2}}&{ - 3{x_3}}\\
{5{x_1}}&{ - 2{x_2}}&{7{x_3}}
\end{array}} \right) = \left( {\begin{array}{*{20}{c}}
0\\
0\\
0
\end{array}} \right) \Rightarrow \left. {\begin{array}{*{20}{c}}
{3{x_1} + {x_2} + 2{x_3} = 0}\\
{{x_1} + 4{x_2} - 3{x_3} = 0}\\
{5{x_1} - 2{x_2} + 7{x_3} = 0}
\end{array}} \right\}\]\\\\

Επομένως το σύστημα αυτό θα μπορεί να λυθεί με τη μέθοδο \textlatin{Cramer} μιας και ενδεικνύετε για την επίλυση τετραγωνικών συστηματων (3Χ3) αλλα και με άλλους τρόπους, όπως τη μέθοδο \textlatin{Gauss-Jordan}
\end{enumerate}

\end{document}